%%This is a very basic article template.
%%There is just one section and two subsections.
\documentclass{article}

%%%%%%%%%%%%%%%%%%%%%%%%%%%%%%%%%%%%%%%%%%%%%%%%%%%%%%%%%%%%%%%%%%%%%
% header, footers for review
\usepackage{fancyhdr}
\pagestyle{fancy}

%\chead{submitted as a relationship report to be included in the Special
%Collection ``Formal Relationships'' edited by Joshua R. Goldstein and James W.
%Vaupel}
%\lfoot{Submission \#2518 for review}
%\rfoot{}
\fancyhf{}
\fancyhead[C]{\footnotesize{submitted as a relationship report to be included in the
Special Collection ``Formal Relationships'' edited by Joshua R. Goldstein and James W.
Vaupel}}
\fancyfoot[R]{\footnotesize{\bfseries{page \thepage}}}
\fancyfoot[L]{\footnotesize\bfseries{Submission \#2518, revised}}
\renewcommand{\headrulewidth}{0pt}
\renewcommand{\footrulewidth}{0pt}
%\setlength\headheight{117.89105pt}
\setcounter{page}{2}
\pagestyle{fancy}

\makeatletter
\let\ps@plain\ps@fancy 
\makeatother

%%%%%%%%%%%%%%%%%%%%%%%%%%%%%%%%%%%%%%%%%%%%%%%%%%%%%%%%%%%%%%
\usepackage{amsmath}
\usepackage{caption}
\usepackage{placeins}
\usepackage{graphicx}
\usepackage{natbib}
\usepackage{setspace}
\doublespacing
\bibpunct{(}{)}{,}{a}{}{;} 
\usepackage{url}
% for the d in integrals
\newcommand{\dd}{\; \mathrm{d}}
% end preamble.

\begin{document}

\title{The force of mortality by life lived is the force of increment by life
left in stationary populations}
\author{Author Redacted}
\maketitle
\section*{Structured abstract}

\subsection*{Background}
The age distribution and remaining lifespan distribution are identical in
stationary populations. The lifetable survival function is proportional to the
age distribution in stationary populations.

\subsection*{Objective}
We provide an alernative interpretation of the lifetable when viewed by
remaining years of life.

\subsection*{Conclusions}
The functions describing the mortality of birth cohorts over age are identical
to the functions describing the growth of death cohorts as time to death decreases in stationary populations.
\vspace{2cm}
\section*{Relationship}

Age can be defined as chronological, $a$ (time since birth), or thanatological,
$y$ (time until death). In a stationary population, decrement over
chronological age, as described by the lifetable, is
identical to increment over thanatological age. Define the chronological force
of mortality, $\mu(a)$, and the thanatological force of increment,
$\mu^\star(y)$. We have:
\begin{equation}
\mu(a) = \mu^\star(y)\quad\quad\text{for~}a=y\quad \text{.}
\end{equation}

\section*{Proof}

Chronological age structure equals thanatological age structure in a stationary
population \citep{brouard1989mouvements,vaupel2009life,rao2014generalization}:

\begin{equation}
\label{eq:vaupel}
c(a) = g(a) \quad\quad \text{,}
\end{equation}
where $c(a)$ is the stable chronological age structure expressed as a
proportion, and $g(a)$ is the proportion of the population with $a$ remaining
years of life. Using $a$ to index chronological age and $y$ to index
thanatological age (time left until death), equation \eqref{eq:vaupel} is the
same as:

\begin{equation}
\label{eq:equivalent}
c(a) = c^\star (y) \quad \text{for}\quad a = y \quad\quad\text{,}
\end{equation}
where $c^\star (y)$ refers to the stable thanatological age
structure. $c(a)$ is proportional to the survival
function, $l(a)$, which means that $c^\star(y)$ is proportional to some
thanatological function, $l^\star (y)$, a stationary thanatological cumulative
\textit{increment} function. To untangle how this is so, it helps to be
explicit about what $l(a)$ represents. A birth cohort is a
group of individuals that are born at the same point in time and experience attrition over age until extinction, as described by $l(a)$. $l(a)$ is also
the sum of all deaths to the birth cohort at ages greater than $a$, and can be
thought of as the de-accumulation of the stock of future deaths over age. 

\begin{equation}
\label{eq:la}
l(a) = \int _{0}^{\omega-a} f(a+t) \dd t \quad \text{,}
\end{equation}

\noindent where $f(a)$ is the lifetable density function, often denoted by
$d(a)$. $f(a)$ gives the probability that a member of the birth cohort born in year $t$
will die in the year $t+a$. Since $t+a$ refers to the future,
let us switch to index $y$, for thanatological age. Under
stationarity, $f(y)$ is also the probability at birth that an individual born
$y$ years ago will die in present year $t$. The deaths that occur together in
year $t$ comprise a \textit{death cohort}. In year, $t-1$, $l(0)\cdot f(1)$ births increment to the year $t$ death cohort in the stationary population. Death cohorts grow monotonically in this way, starting
with a few members that will enjoy the maximum attainable lifespan, $ y =
\omega$, i.e., born $\omega$ years ago in year $t-\omega$ and then expiring in
year $t$. New births accumulate into a given death cohort as $y$ decreases from
$\omega$ toward thanatological age 0. From the vantage point of year
$t$, we define the year $t+y$ death cohort ($y$ years in the future), $l^\star
(y)$ as the members born up to year $t$ that will die in exactly $y$ years. 

\begin{equation}
\label{eq:lstar}
l^\star (y) = \int _{-(\omega - y)}^0 f(y-t) \dd t \quad \text{.}
\end{equation}

\noindent Extinction of the death cohort is simultaneous upon reaching
thanatological age 0, $y$ years in the future. Equation~\ref{eq:lstar} is equal
to \ref{eq:la}:

\begin{equation}
\begin{split}
l^\star (y) = \int_{-(\omega-y)}^0 f(y-t)\dd t = \int _{-\omega}^{-y}
f(-t)\dd t = \int _y ^\omega f(t) \dd t = l(y) \quad \text{.}
\end{split}
\end{equation}

The lifetable deaths distribution is under
this perspective a distribution of the births to a death cohort, when read
from the highest to the lowest chronological ages. The rate at which new births
accumulate to death cohorts over thanatological age, $\mu^\star (y)$, is given
by
\begin{align}
\mu ^\star (y) &= \frac{-l^{\star^\prime} (y)}{l^\star (y)} \quad \text{,}
               \intertext{and since $l(a) = l^\star(y)$ for $a = y$,}
               &= \frac{-l^\prime (a)}{l(a)} \notag\\
               &= \mu (a) \quad \text{.}
\end{align}

The rate of birth accumulation into death cohorts over thanatological age,
$\mu^\star (y)$, is equal to the rate of attrition of birth cohorts
over chronological age, $\mu (a)$ in stationary populations. The stock of deaths in
the future for chronological age $a$, $l(a)$, is symmetrically a description of births in the
past when structured by thanatological age, $l^\star(y)$. The remaining
lifetable columns are subject to similar reinterpretations when viewed under
thanatological age. The main contribution of this relationship is to point out
the symmetry of chronological decrement and thanatological renewal processes in
the case of stationary populations.

\section*{History and related results}

The term ``thanatological age'' has not previously appeared in the
literature, and was coined by Ken Wachter sometime prior to 2001. Thanatos was
the Greek god of death, which is used as the reference point for age under this
perspective. Explicit decompositions of chronological
age groups into remaining lifespan classes is to our knowledge only found in
\citet{brouard1986structure}, who redistributed
population pyramids by remaining years of life, and
\citet{miller2001increasing}, who examined medicare expenditure as a function of time until death using the
same lifetable identities. Equivalence of chronological and
thanatological age structures in stationarity, also known as Carey's equality,
is proven by \citet{brouard1989mouvements}, again in
 \citet{vaupel2009life},\footnote{\citet{vaupel2009life} added detail to the
 more summary result reported in \citet{goldstein2009life}, which showed that
 the mean (or total) remaining lifespan is equal to mean (or total) life lived in the stationary population.} and in more detail by \citet{rao2014generalization}.
These latter two papers and \citet{goldstein2012historical} describe more of the
lineage of the result presented here. 

The estimation of remaining lifespan has motivated demography since
the invention of the lifetable, but this notion is typically dealt with as a
mean (expectancy). Age-specific remaining lifespan distributions are not often
used to explicitly decompose demographic quantities, such as population counts. Using
the above identities, one may estimate the total population with $y$
remaining years of life, $P(y)$ directly
\begin{equation}
\label{eq:getpy}
P(y) = \int _0 ^\infty P(a)\mu(a+y)\frac{l(a+y)}{l(a)} \dd a \quad \text{,}
\end{equation}
\noindent as \citet{brouard1986structure} did, and so approximate a population's
thanatological age structure according to some mortality assumptions. This result differs from the more common approach
of summing the population within age classes bounded by some values interpolated
along the remaining life expectancy function (e.g.
\citet{sanderson2005average,sanderson2007new,sanderson2010remeasuring}, who
follow the line of \citet{hersch1944demographie} and \citet{ryder1975notes}).

In the present case of a stationary population, one decomposes back to birth
cohorts in much the same way as is evident from \eqref{eq:getpy}. The
probability that a member of death cohort $y'$ was born $a'$ years ago is equal to the probability that a member of birth cohort $a$ will die $y$ years in the future when $a = y'$ and $y = a'$.

\begin{align}
f(y | a) &= f(a' | y') \quad \text{for } a = y' \;\text{, } y = a'\\
&= \mu (a+y)\frac{l(a+y)}{l(a)}\\
&= \mu^\star (a'+y')\frac{l^\star(a'+y')}{l^\star(y')} \label{eq:redundant}
\end{align}

Equation \eqref{eq:redundant} seems redundant, but is less obvious when put into
words: The probability of dying $y$ years in the future given survival to chronological age
$a$ is the probability of surviving to chronological age $a+y$ given survival to
age $a$ times the force of mortality at age $a+y$, $\mu(a+y)$. Viewed thanatologically, the
probability of having been born $a'$ years in the past given that one has $y'$
remaining years of life is equal to the probability that someone in the
$y'$ death cohort was born more than $a'$ years ago given that they have already
been born times the force of increment at thanatological age $a'+y'$, $\mu
^\star (a'+y')$.

Similarly, taking $f(y|a)$ as the conditional density of remaining
lifespans, one may calculate the variance of age-specific remaining life
spans (thanatological age), where $e(a)$ is remaining life expectancy at age
$a$:
\begin{align}
e(a) =& \frac{\int _{y=0} ^\infty l(a+y) \dd y}{l(a)}
\intertext{The variance of $y$ given survival to age $a$ is}
\sigma ^2 (y|a) =& \int _{y=0}^\infty \left( e(a) - y\right)^2 f(y|a) \dd y
\quad \text{,}
\end{align}
\noindent and this function will have some non-monotonic pattern over age
(in human populations) that remains to be explored.\footnote{This
definiton works out to be the same as \citet{chiang1984life}, Chapter 10,
Equation 6.10.} In the reliability literature, $\sigma ^2(y|a)$ is called the variance residual life function (VRLF), and its properties have been described for various common distributions
\citep[see for example][]{gupta2006variance}.

The thanatological age perspective only offers the kind of profile symmetry
presented in this paper for the theoretical case of stationary populations. For
changing populations, the chronological and thanatological age perspectives
typically offer different profiles of the same phenomena, due to changes in
lifespan distributions and fluctuations in the birth flow, and therefore offer
complementary information on population structure.

\section*{Applications}
Thanatological age structure can be applied to stable populations (subject to a
growth rate, $r$), though we leave the description of a thanatological
renewal model for future work. Thanatological age equates individuals that share
a common terminal state rather than a common origin state. In the present
relationship, this is the absorbing state of death, but the method generalizes to any terminal state
or lifecourse transition that can be modeled using lifetable techniques. Potential area
applications that may gain insights using such remaining-time methods
include morbidity, disability, late-life savings and investment behavior, or
perhaps time to birth, menopause, retirement, or graduation. Glacial or open
ice pack, old growth forest, and prison populations are other examples of
aggregates for which remaining time structure is inherently of equal or greater
interest than time passed. Populations of fixed or controlled size, or where
entries are largely a function of exits are also prime candidates for analysis under some
analogy to thanatological age. Examples of such populations include professional
athletes in leagues, tenured professors, company directors, and vehicle fleets.

As a specific example, the question of morbidity compression has often been
posed as a matter of comparing age of onset with remaining life expectancy
\citep[e.g., ][]{fries2002aging,fries2003measuring}.\footnote{In the health
literature, one often sees $e(65)$ compared with average disability levels above age 65.} In
retrospect (within death cohorts), one can analytically separate between changes
in lifespan and changes in morbidity as a function of time until death. Cross-tabulations of morbidity by chronological and thanatological age for two birth cohorts would allow the
researcher to estimate changes in the morbidity profile over time free from
distortion due to changes in the lifespan distribution, and so directly settle
the question of morbidity compression.

\bibliographystyle{plainnat}
  \bibliography{bibliography}   % Use the BibTeX file ``References.bib''.


\end{document}
% file.copy(from = "/home/triffe/git/LifetableIncrement/LifetableIncrement/LifetableIncrement.pdf", to = "/home/triffe/git/LifetableIncrement/LifetableIncrement/2518.pdf",overwrite= TRUE)
