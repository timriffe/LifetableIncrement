%%This is a very basic article template.
%%There is just one section and two subsections.
\documentclass{article}

%%%%%%%%%%%%%%%%%%%%%%%%%%%%%%%%%%%%%%%%%%%%%%%%%%%%%%%%%%%%%%%%%%%%%
% header, footers for review
\usepackage{fancyhdr}
\pagestyle{fancy}

%\chead{submitted as a relationship report to be included in the Special
%Collection ``Formal Relationships'' edited by Joshua R. Goldstein and James W.
%Vaupel}
%\lfoot{Submission \#2518 for review}
%\rfoot{}
\fancyhf{}
\fancyhead[C]{\footnotesize{submitted as a relationship report to be included in the
Special Collection ``Formal Relationships'' edited by Joshua R. Goldstein and James W.
Vaupel}}
\fancyfoot[R]{\footnotesize{\bfseries{page \thepage}}}
\fancyfoot[L]{\footnotesize\bfseries{Submission \#2518 for review}}
\renewcommand{\headrulewidth}{0pt}
\renewcommand{\footrulewidth}{0pt}
%\setlength\headheight{117.89105pt}
\setcounter{page}{2}
\pagestyle{fancy}

\makeatletter
\let\ps@plain\ps@fancy 
\makeatother

%%%%%%%%%%%%%%%%%%%%%%%%%%%%%%%%%%%%%%%%%%%%%%%%%%%%%%%%%%%%%%
\usepackage{amsmath}
\usepackage{caption}
\usepackage{placeins}
\usepackage{graphicx}
\usepackage{natbib}
\usepackage{setspace}
\doublespacing
\bibpunct{(}{)}{,}{a}{}{;} 
\usepackage{url}
% for the d in integrals
\newcommand{\dd}{\; \mathrm{d}}
% end preamble.

\begin{document}

\title{The force of mortality by life lived is the force of increment by life
left in stationary populations}
\author{Author Redacted}
\maketitle
\section*{Structured abstract}

\subsection*{Background}
The age distribution and remaining lifespan distribution are identical in
stationary populations. The lifetable survival function is proportional to the
age distribution in stationary populations.

\subsection*{Objective}
We provide an alernative interpretation of the lifetable when viewed by
remaining years of life.

\subsection*{Conclusions}
The functions describing the mortality of birth cohorts over age are identical
to the functions describing the growth of death cohorts as time to death decreases in stationary populations.
\vspace{2cm}
\section*{Relationship}

Age can be defined as chronological (time since birth) or thanatological (time
until death). In a stationary population, decrement over
chronological age, as described by the lifetable, is
identical to increment over thanatological age.

\section*{Proof}

Chronological age structure equals thanatological age structure in a stationary
population \citep{brouard1989mouvements,vaupel2009life}:

\begin{equation}
\label{eq:vaupel}
c(a) = g(a) \quad\quad \text{,}
\end{equation}
where $c(a)$ is the stable chronological age structure expressed as a
proportion, and $g(a)$ is proportion of the population with $a$ remaining years
of life. Using $a$ to index chronological age and $y$ to index
thanatological age, the amount of time left until death. Equation
\eqref{eq:vaupel} is the same as:

\begin{equation}
\label{eq:equivalent}
c(a) = c^\star (y) \quad \text{for}\quad a = y \quad\quad\text{,}
\end{equation}
where $c^\star (y)$ refers to the stable thanatological age
structure. $c(a)$ is proportional to the survival
function, $l(a)$, which means that $l(a)$ is proportional to some
thanatological function, $l^\star (y)$, a stationary thanatological cumulative
\textit{increment} function. To untangle how this is so, it helps to be
explicit about what $l(a)$ represents. A birth cohort is a
group of individuals that are born at the same point in time and experience attrition over age until extinction, as described by $l(a)$. $l(a)$ is also
the sum of all deaths to the birth cohort at ages greater than $a$, and can be
thought of as the de-accumulation of the stock of future deaths over age. 

\begin{equation}
l(a) = \int _{0}^\infty f(a+y) \dd y \quad \text{,}
\end{equation}

\noindent where $f(a)$ is the lifetable density function, often denoted by
$d(a)$. $f(a)$ gives the probability that a member of the birth cohort born in year $t$
will die in the year $t+a$. The deaths that will occur together in year
$t+a$ comprise a \textit{death cohort}. In the next year $t+1$,
$l(0)\cdot f(a-1)$ births increment to the $t+a$ death cohort in the stationary
population. Death cohorts grow monotonically in this way, starting with a few
members that will enjoy the maximum attainable lifespan, $ y = \omega$, and
accumulating new members as time approaches $t_0+\omega$ and $y$ decreases
toward 0:

\begin{equation}
l^\star (y) = \int _\omega^0 f(y) \dd y \quad \text{.}
\end{equation}

\noindent Extinction of the death cohort is simultaneous upon reaching
thanatological age 0. The lifetable deaths distribution is under this
perspective a distribution of the births to a death cohort, when read
from the highest to the lowest ages. The rate at which new births accumulate to
death cohorts over thanatological age, $\mu^\star (y)$ is given by
\begin{align}
\mu ^\star (y) &= \frac{-l^{\star^\prime} (y)}{l^\star (y)} \quad \text{,}
               \intertext{and since $l(a) = l^\star{a}$:}
               &= \frac{-l^\prime (y)}{l(y)} \notag\\
               &= \mu (y)
\end{align}

The rate of birth accumulation into death cohorts over thanatological age is
equal to the rate of death attrition of birth cohorts over chronological age in
stationary populations. The stock of deaths in the future for
chronological age $a$, $l(a)$, is symmetrically a description of births in the
past when structured by thanatological age, $l^\star(y)$. The remaining
lifetable columns are subject to similar reinterpretations when viewed under
thanatological age.

\section*{History and related results}

The term ``thanatological age'' has not previously appeared in the
literature, and was coined by Ken Wachter sometime prior to 2001. Thanatos was
the Greek god of death, which is used as the reference point for age under this
perspective. Explicit decompositions of chronological
age groups into remaining lifespan classes is to our knowledge only found in
\citet{brouard1986structure}, who redistributed
population pyramids by remaining years of life, and
\citet{miller2001increasing}, who examined medicare expenditure as a function of time until death using the
same lifetable identities. Equivalence of chronological and
thanatological age structures in stationarity, also known as Carey's equality,
is proven by \citet{brouard1989mouvements} and again in
 \citet{vaupel2009life}, which added detail to the more summary result reported
 in \citet{goldstein2009life}, that mean (or total) remaining lifespan is equal to mean (or total) life lived in the stationary population.
These latter two papers and \citet{goldstein2012historical} describe more of the
lineage of result presented here. 

The estimation of remaining lifespan has motivated demography since
the invention of the lifetable, but this notion is typically dealt with as a
mean (expectancy). Age-specific remaining lifespan distributions are not often
used to explicitly decompose demographic quantities, such as population counts. Using
the above identities, one may estimate the total population with $y$
remaining years of life, $P(y)$ directly
\begin{equation}
\label{eq:getpy}
P(y) = \int _0 ^\infty P(a)\mu(a+y)\frac{l(a+y)}{l(a)} \dd a \quad \text{,}
\end{equation}
\noindent as \citet{brouard1986structure} did, and so approximate a population's
thanatological age structure according to some mortality assumptions. This result differs from the more common approach
of summing the population within age classes bounded by some values interpolated
along the remaining life expectancy function (e.g.
\citet{sanderson2005average,sanderson2007new,sanderson2010remeasuring}, who
follow the line of \citet{hersch1944demographie} and \citet{ryder1975notes}).

In the present case of a stationary population, one decomposes back to birth
cohorts in much the same way as is evident from \eqref{eq:getpy}. The
probability that a member of death cohort $y'$ was born $a'$ years ago is equal to the probability that a member of birth cohort $a$ will die $y$ years in the future when $a = y'$ and $y = a'$.

\begin{align}
f(y | a) &= f(a' | y') \quad \text{for } a = y' \;\text{, } y = a'\\
&= \mu (a+y)\frac{l(a+y)}{l(a)}\\
&= \mu^\star (a'+y')\frac{l^\star(a'+y')}{l^\star(y')} \label{eq:redundant}
\end{align}

Equation \eqref{eq:redundant} seems redundant, but is less obvious when put into
words: The probability of dying $y$ years in the future given survival to chronological age
$a$ is the probability of surviving to chronological age $a+y$ given survival to
age $a$ times the force of mortality at age $a+y$, $\mu(a+y)$. Viewed thanatologically, the
probability of having been born $a'$ years in the past given that one has $y'$
remaining years of life is equal to the probability that someone in the
$y'$ death cohort was born more than $a'$ years ago given that they have already
been born times the force of increment at thanatological age $a'+y'$, $\mu
^\star (a'+y')$.

Similarly, taking $f(y|a)$ as the conditional density of remaining
lifespans, one may calculate the variance of age-specific remaining life
spans (thanatological age), where $e(a)$ is remaining life expectancy at age
$a$:
\begin{align}
e(a) =& \frac{\int _{y=0} ^\infty l(a+y) \dd y}{l(a)}
\intertext{The variance of $y$ given survival to age $a$ is}
\sigma ^2 (y|a) =& \int _{y=0}^\infty \left( e(a) - y\right)^2 f(y|a) \dd y
\quad \text{,}
\end{align}
\noindent and this function will have some non-monotonic pattern over age
(in human populations) that remains to be explored. In the reliability
literature, $\sigma ^2(y|a)$ is called the variance residual life function
(VRLF), and its properties have been described for various common distributions
\citep[see for example][]{gupta2006variance}.

\section*{Applications}
Thanatological age structure can be applied to stable populations (subject to a
growth rate, $r$), though we leave the description of a thanatological
renewal model for future work. Thanatological age equates individuals that share
a common terminal state rather than a common origin state. Here this is
the absorbing state of death, but the method generalizes to any terminal state
or lifecourse transition that can be modeled using lifetable techniques. Potential area
applications that may gain insights using such remaining-time methods
include morbidity, disability, late-life savings and investment behavior, or perhaps time to birth, menopause, retirement, or graduation.

\bibliographystyle{plainnat}
  \bibliography{bibliography}   % Use the BibTeX file ``References.bib''.


\end{document}
% file.copy(from = "/home/triffe/git/LifetableIncrement/LifetableIncrement/LifetableIncrement.pdf", to = "/home/triffe/git/LifetableIncrement/LifetableIncrement/2518.pdf",overwrite= TRUE)
