

\documentclass{article}

\usepackage{amsmath}
\usepackage{natbib}
\bibpunct{(}{)}{,}{a}{}{;} 
\newcommand{\dd}{\; \mathrm{d}}
\usepackage{url}


\begin{document}
\title{Reply to Reviewers of submission 2518: ``The force of mortality by life
lived is the force of increment by life left in stationary populations''}
\author{Author Redacted}
\maketitle
Thanks for the useful comments, which improved the paper. I
hope that concerns are now met with the current revision.

Reviewer A gave math and notation suggestions, which were correct and which I
heeded. I think the primary concern is mended by changing former equation (4)
(now equation (5)) to:
\begin{equation}
l^\star (y) = \int _\omega^y f(t) \dd t \notag \quad \text{.} 
\end{equation}

Reviewer B suggested I add detail to the applications section. Admittedly, these
applications have to do with the thanatological age perspective in general
rather than the neatness of symmetry with respect to chronological decrement and
thanatological renewal within the lifetable. I hope my suggested applications
are an acceptable ersatz. The specific case I now add is for the measurement of
morbidity patterns in order to approach the question of whether or not there
has been compression of morbidity. This question has been asked many times, and
it appears that the usual approach is to simply compare $e(65)$ with average
disability rates at ages $65+$. That is a very rough gauge. Measuring morbidity
in thanatological age would help disentangle morbidity change from lifespan change,
which is what we really want in order to say something about compression.

Reviewer C likened the thanatological perspective change to viewing the contents
of a room from above. This is a good analogy in the case of stationary
populations, but note that it only works for the marginal profile of lifetable
functions. The perspective change is more complex than that, bearing in mind
that the functions are decomposable into chronological and thanatological age
(described for the case of the pyramid). For an image of such heterogeneity, see
the colored images in this abstract to PAA, 2014: \url{http://paa2014.princeton.edu/papers/141036}
(before I knew about \citep{brouard1986structure}). I don't think an image is
necessary in this paper, to appreciate this, though, but I will oblige if the
wish is echoed further. I added a line at the end of the proof section ``The
main contribution of this relationship is to point out the symmery of chronological decrement and thanatological renewal processes in
the case of stationary populations.'' This contribution isn't revolutionary or
of great use, but the lifetable, stationary and stable populations, and the
renewal model stand at the core of demography, which ought to be
solid. I think tiny observations about them are worthy of publishing, and moreso
in this series, which values aesthetics. 

Note that the stationary age structure finding, on which this result builds, has
appeared in the literatute again, since this paper was submitted
\citep{rao2014generalization}. These and previous authors were apparently also
not aware of Brouard's contributions. I hope to subtly set the record straight.

\bibliographystyle{plainnat}
  \bibliography{bibliography} 

\end{document}

