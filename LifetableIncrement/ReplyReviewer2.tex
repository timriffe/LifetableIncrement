\documentclass{article}

\usepackage{amsmath}
\usepackage{natbib}
\bibpunct{(}{)}{,}{a}{}{;} 
\newcommand{\dd}{\; \mathrm{d}}
\usepackage{url}


\begin{document}
\title{Second reply to reviewers of submission 2518: ``The force of mortality by
life lived is the force of increment by life left in stationary populations''}
\author{Author Redacted}
\maketitle

My thanks to the reviewers for the corrections and constructive suggestions.
Reviewer A found an error that owes to my own misunderstanding of integrals,
which until now I've naively understood as perfectly analogous to discrete sums.
In other words, the old equation (5) works when you draw it on paper, but now I
understand why it does not work with calculus notation. Reviewer A also
provided a solution that saves the argument, and for which I am grateful.

I've only made yet another innocuous change to both my own equation (4) and
reviewer A's suggestion for equation (5):

Equation 4 Was:
\begin{equation}
\label{eq:la}
l(a) = \int _{0}^{\omega} f(a+t) \dd t \quad \text{,}
\end{equation}
Now:
\begin{equation}
\label{eq:la}
l(a) = \int _{0}^{\omega-a} f(a+t) \dd t \quad \text{.}
\end{equation}
This is just aesthetic, I suppose, but really it's the sum ($a+t$)
that takes a maximum value of $\omega$. It's an innocuous change since higher
values are all zero. I made this change because it helped me work through reviewer A's
equation of $l^\star (y)$ with $l(a)$ when $a=y$, so I changed the reviewers
alternative definition of $l^\star (y)$ to:
\begin{equation}
\label{eq:lstar}
l^\star (y) = \int _{-(\omega - y)}^0 f(y-t) \dd t \quad \text{,}
\end{equation}
and everything works out to be the same by my reckoning.

The text between equations 4 and 5 has also been substantially changed to
better reflect the new formulation of $l^\star (y)$, and I hope it is clear.

I also changed the indexing of the final result, and included ``for $a=y$'', as
requested by the reviewer. You're right, it's easier to understand that way.
-----------------------------

Reviewer C looked for the old Brouard material. This is a link a a 1989 version of the Brouard paper with a version of Carey's inequality in it:

\url{http://sauvy.ined.fr/brouard/enseignements/iford/mouvementetmodeles.pdf}

The juicy part is section 4.2.2, ``Pyramide des ann\'{e}es a vivre'', which is
couched in terms of the stationary population. I do think that he will have
first written this down earlier, since he cites the relationship in
\citet{brouard1986structure}, which is better known for having introduced the
measure 'CAL'.

Here's a link to the 1986 paper: 
\url{http://www.persee.fr/web/revues/home/prescript/article/espos_0755-7809_1986_num_4_2_1120}

The key sentence in the  1986 paper, which led me to discover the other more
obscure source, is at the bottom of page 159, and continues into page 161 (pg
160 is a figure page)- he states Careys equality in plain French and cites a 1985 manuscript ``Mouvements et mod\`{e}les de population''
to back it up. I've never been able to locate the 1985 version, but it appears
to be something he maintained over a series of years, and the
first linked manuscript above appears to be from 1989 (so I cite it as such). Looking at the inside title page it says 1986. It appears to have been placed online in 2002.

To settle the date more exactly, one would have to ask N. Brouard when he
thought of it (or where he saw it). I imagine he went to Cameroon in 1985 with a
notebook of concepts to demonstrate, and either he knew it beforehand, or it occurred to him there, but it wasn't, as far as I
can tell, in a distributed print form until 1989. He apparently cited his 1985
workshop materials, though I could be mistaken. I've not seen a hard copy.

Regarding \citet{rao2014generalization}, the citation seems OK to me, but it was
easy to miss since the bibliography is alphabetical by last name but printed
starting with first name (Arni). I'll reformat as the journal requires.
-----------------------------------
Finally, I added a reference to \citet{chiang1984life}, who provides an
alternative but equivalent expression of $\sigma^2(y|a)$ (in the History \&
Related Results section), and I also fixed some minor typos here and there.

Thanks for the careful reviewing.

\bibliographystyle{plainnat}
  \bibliography{bibliography} 

\end{document}
